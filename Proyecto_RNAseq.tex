% Options for packages loaded elsewhere
\PassOptionsToPackage{unicode}{hyperref}
\PassOptionsToPackage{hyphens}{url}
%
\documentclass[
]{article}
\usepackage{amsmath,amssymb}
\usepackage{iftex}
\ifPDFTeX
  \usepackage[T1]{fontenc}
  \usepackage[utf8]{inputenc}
  \usepackage{textcomp} % provide euro and other symbols
\else % if luatex or xetex
  \usepackage{unicode-math} % this also loads fontspec
  \defaultfontfeatures{Scale=MatchLowercase}
  \defaultfontfeatures[\rmfamily]{Ligatures=TeX,Scale=1}
\fi
\usepackage{lmodern}
\ifPDFTeX\else
  % xetex/luatex font selection
\fi
% Use upquote if available, for straight quotes in verbatim environments
\IfFileExists{upquote.sty}{\usepackage{upquote}}{}
\IfFileExists{microtype.sty}{% use microtype if available
  \usepackage[]{microtype}
  \UseMicrotypeSet[protrusion]{basicmath} % disable protrusion for tt fonts
}{}
\makeatletter
\@ifundefined{KOMAClassName}{% if non-KOMA class
  \IfFileExists{parskip.sty}{%
    \usepackage{parskip}
  }{% else
    \setlength{\parindent}{0pt}
    \setlength{\parskip}{6pt plus 2pt minus 1pt}}
}{% if KOMA class
  \KOMAoptions{parskip=half}}
\makeatother
\usepackage{xcolor}
\usepackage[margin=1in]{geometry}
\usepackage{color}
\usepackage{fancyvrb}
\newcommand{\VerbBar}{|}
\newcommand{\VERB}{\Verb[commandchars=\\\{\}]}
\DefineVerbatimEnvironment{Highlighting}{Verbatim}{commandchars=\\\{\}}
% Add ',fontsize=\small' for more characters per line
\usepackage{framed}
\definecolor{shadecolor}{RGB}{248,248,248}
\newenvironment{Shaded}{\begin{snugshade}}{\end{snugshade}}
\newcommand{\AlertTok}[1]{\textcolor[rgb]{0.94,0.16,0.16}{#1}}
\newcommand{\AnnotationTok}[1]{\textcolor[rgb]{0.56,0.35,0.01}{\textbf{\textit{#1}}}}
\newcommand{\AttributeTok}[1]{\textcolor[rgb]{0.13,0.29,0.53}{#1}}
\newcommand{\BaseNTok}[1]{\textcolor[rgb]{0.00,0.00,0.81}{#1}}
\newcommand{\BuiltInTok}[1]{#1}
\newcommand{\CharTok}[1]{\textcolor[rgb]{0.31,0.60,0.02}{#1}}
\newcommand{\CommentTok}[1]{\textcolor[rgb]{0.56,0.35,0.01}{\textit{#1}}}
\newcommand{\CommentVarTok}[1]{\textcolor[rgb]{0.56,0.35,0.01}{\textbf{\textit{#1}}}}
\newcommand{\ConstantTok}[1]{\textcolor[rgb]{0.56,0.35,0.01}{#1}}
\newcommand{\ControlFlowTok}[1]{\textcolor[rgb]{0.13,0.29,0.53}{\textbf{#1}}}
\newcommand{\DataTypeTok}[1]{\textcolor[rgb]{0.13,0.29,0.53}{#1}}
\newcommand{\DecValTok}[1]{\textcolor[rgb]{0.00,0.00,0.81}{#1}}
\newcommand{\DocumentationTok}[1]{\textcolor[rgb]{0.56,0.35,0.01}{\textbf{\textit{#1}}}}
\newcommand{\ErrorTok}[1]{\textcolor[rgb]{0.64,0.00,0.00}{\textbf{#1}}}
\newcommand{\ExtensionTok}[1]{#1}
\newcommand{\FloatTok}[1]{\textcolor[rgb]{0.00,0.00,0.81}{#1}}
\newcommand{\FunctionTok}[1]{\textcolor[rgb]{0.13,0.29,0.53}{\textbf{#1}}}
\newcommand{\ImportTok}[1]{#1}
\newcommand{\InformationTok}[1]{\textcolor[rgb]{0.56,0.35,0.01}{\textbf{\textit{#1}}}}
\newcommand{\KeywordTok}[1]{\textcolor[rgb]{0.13,0.29,0.53}{\textbf{#1}}}
\newcommand{\NormalTok}[1]{#1}
\newcommand{\OperatorTok}[1]{\textcolor[rgb]{0.81,0.36,0.00}{\textbf{#1}}}
\newcommand{\OtherTok}[1]{\textcolor[rgb]{0.56,0.35,0.01}{#1}}
\newcommand{\PreprocessorTok}[1]{\textcolor[rgb]{0.56,0.35,0.01}{\textit{#1}}}
\newcommand{\RegionMarkerTok}[1]{#1}
\newcommand{\SpecialCharTok}[1]{\textcolor[rgb]{0.81,0.36,0.00}{\textbf{#1}}}
\newcommand{\SpecialStringTok}[1]{\textcolor[rgb]{0.31,0.60,0.02}{#1}}
\newcommand{\StringTok}[1]{\textcolor[rgb]{0.31,0.60,0.02}{#1}}
\newcommand{\VariableTok}[1]{\textcolor[rgb]{0.00,0.00,0.00}{#1}}
\newcommand{\VerbatimStringTok}[1]{\textcolor[rgb]{0.31,0.60,0.02}{#1}}
\newcommand{\WarningTok}[1]{\textcolor[rgb]{0.56,0.35,0.01}{\textbf{\textit{#1}}}}
\usepackage{graphicx}
\makeatletter
\def\maxwidth{\ifdim\Gin@nat@width>\linewidth\linewidth\else\Gin@nat@width\fi}
\def\maxheight{\ifdim\Gin@nat@height>\textheight\textheight\else\Gin@nat@height\fi}
\makeatother
% Scale images if necessary, so that they will not overflow the page
% margins by default, and it is still possible to overwrite the defaults
% using explicit options in \includegraphics[width, height, ...]{}
\setkeys{Gin}{width=\maxwidth,height=\maxheight,keepaspectratio}
% Set default figure placement to htbp
\makeatletter
\def\fps@figure{htbp}
\makeatother
\setlength{\emergencystretch}{3em} % prevent overfull lines
\providecommand{\tightlist}{%
  \setlength{\itemsep}{0pt}\setlength{\parskip}{0pt}}
\setcounter{secnumdepth}{-\maxdimen} % remove section numbering
\ifLuaTeX
  \usepackage{selnolig}  % disable illegal ligatures
\fi
\usepackage{bookmark}
\IfFileExists{xurl.sty}{\usepackage{xurl}}{} % add URL line breaks if available
\urlstyle{same}
\hypersetup{
  pdftitle={Proyecto\_RNAseq},
  pdfauthor={Karla Ximena González Platas},
  hidelinks,
  pdfcreator={LaTeX via pandoc}}

\title{Proyecto\_RNAseq}
\usepackage{etoolbox}
\makeatletter
\providecommand{\subtitle}[1]{% add subtitle to \maketitle
  \apptocmd{\@title}{\par {\large #1 \par}}{}{}
}
\makeatother
\subtitle{Análisis de Expresión Diferencial}
\author{Karla Ximena González Platas}
\date{2025-02-05}

\begin{document}
\maketitle

{
\setcounter{tocdepth}{2}
\tableofcontents
}
\section{Introducción}\label{introducciuxf3n}

\section{Instalar y cargas paquetes}\label{instalar-y-cargas-paquetes}

\begin{Shaded}
\begin{Highlighting}[]
\CommentTok{\# Cargar el paquete de R que incluye a SummarizedExperiment y todas las demás dependencias}
\FunctionTok{library}\NormalTok{(}\StringTok{"recount3"}\NormalTok{)}
\end{Highlighting}
\end{Shaded}

\begin{Shaded}
\begin{Highlighting}[]
\CommentTok{\# Identificar el proyecto a trabajar}
\NormalTok{human\_projects }\OtherTok{\textless{}{-}} \FunctionTok{available\_projects}\NormalTok{()}
\end{Highlighting}
\end{Shaded}

\begin{verbatim}
## 2025-02-04 23:03:03.035188 caching file sra.recount_project.MD.gz.
\end{verbatim}

\begin{verbatim}
## 2025-02-04 23:03:03.754218 caching file gtex.recount_project.MD.gz.
\end{verbatim}

\begin{verbatim}
## 2025-02-04 23:03:04.324096 caching file tcga.recount_project.MD.gz.
\end{verbatim}

\begin{Shaded}
\begin{Highlighting}[]
\FunctionTok{dim}\NormalTok{(human\_projects)}
\end{Highlighting}
\end{Shaded}

\begin{verbatim}
## [1] 8742    6
\end{verbatim}

\begin{Shaded}
\begin{Highlighting}[]
\FunctionTok{head}\NormalTok{(human\_projects)}
\end{Highlighting}
\end{Shaded}

\begin{verbatim}
##     project organism file_source     project_home project_type n_samples
## 1 SRP107565    human         sra data_sources/sra data_sources       216
## 2 SRP149665    human         sra data_sources/sra data_sources         4
## 3 SRP017465    human         sra data_sources/sra data_sources        23
## 4 SRP119165    human         sra data_sources/sra data_sources         6
## 5 SRP133965    human         sra data_sources/sra data_sources        12
## 6 SRP096765    human         sra data_sources/sra data_sources         7
\end{verbatim}

\begin{Shaded}
\begin{Highlighting}[]
\CommentTok{\# Seleccionar un estudio de interés}
\NormalTok{human\_projects[}\DecValTok{57}\NormalTok{, ]}
\end{Highlighting}
\end{Shaded}

\begin{verbatim}
##      project organism file_source     project_home project_type n_samples
## 57 SRP068565    human         sra data_sources/sra data_sources        20
\end{verbatim}

\begin{Shaded}
\begin{Highlighting}[]
\DocumentationTok{\#\# Colocar el ID del proyecto}
\NormalTok{project\_info }\OtherTok{\textless{}{-}} \FunctionTok{subset}\NormalTok{(}
\NormalTok{  human\_projects,}
\NormalTok{  project }\SpecialCharTok{==} \StringTok{"SRP068565"} \SpecialCharTok{\&}\NormalTok{ project\_type }\SpecialCharTok{==} \StringTok{"data\_sources"}
\NormalTok{)}

\NormalTok{project\_info}
\end{Highlighting}
\end{Shaded}

\begin{verbatim}
##      project organism file_source     project_home project_type n_samples
## 57 SRP068565    human         sra data_sources/sra data_sources        20
\end{verbatim}

\begin{Shaded}
\begin{Highlighting}[]
\CommentTok{\# Identificar el proyecto a trabajar}

\CommentTok{\# Obtener la lista de proyectos disponibles en el objeto \textquotesingle{}human\_projects\textquotesingle{}}
\NormalTok{human\_projects }\OtherTok{\textless{}{-}} \FunctionTok{available\_projects}\NormalTok{()}
\end{Highlighting}
\end{Shaded}

\begin{verbatim}
## 2025-02-04 23:03:13.055971 caching file sra.recount_project.MD.gz.
\end{verbatim}

\begin{verbatim}
## 2025-02-04 23:03:13.606616 caching file gtex.recount_project.MD.gz.
\end{verbatim}

\begin{verbatim}
## 2025-02-04 23:03:14.152815 caching file tcga.recount_project.MD.gz.
\end{verbatim}

\begin{Shaded}
\begin{Highlighting}[]
\CommentTok{\# Ver los proyectos disponibles}
\FunctionTok{dim}\NormalTok{(human\_projects)}
\end{Highlighting}
\end{Shaded}

\begin{verbatim}
## [1] 8742    6
\end{verbatim}

\begin{Shaded}
\begin{Highlighting}[]
\CommentTok{\# Mostrar las primeras filas para inspeccionar su estructura y contenido}

\FunctionTok{head}\NormalTok{(human\_projects)}
\end{Highlighting}
\end{Shaded}

\begin{verbatim}
##     project organism file_source     project_home project_type n_samples
## 1 SRP107565    human         sra data_sources/sra data_sources       216
## 2 SRP149665    human         sra data_sources/sra data_sources         4
## 3 SRP017465    human         sra data_sources/sra data_sources        23
## 4 SRP119165    human         sra data_sources/sra data_sources         6
## 5 SRP133965    human         sra data_sources/sra data_sources        12
## 6 SRP096765    human         sra data_sources/sra data_sources         7
\end{verbatim}

\begin{Shaded}
\begin{Highlighting}[]
\CommentTok{\# Seleccionar un estudio de interés}
\NormalTok{human\_projects[}\DecValTok{57}\NormalTok{, ]}
\end{Highlighting}
\end{Shaded}

\begin{verbatim}
##      project organism file_source     project_home project_type n_samples
## 57 SRP068565    human         sra data_sources/sra data_sources        20
\end{verbatim}

\begin{Shaded}
\begin{Highlighting}[]
\CommentTok{\# Filtrar el dataframe para seleccionar un proyecto específico basado en su ID y tipo}

\NormalTok{project\_info }\OtherTok{\textless{}{-}} \FunctionTok{subset}\NormalTok{(}
\NormalTok{  human\_projects,}
\NormalTok{  project }\SpecialCharTok{==} \StringTok{"SRP068565"} \SpecialCharTok{\&}\NormalTok{ project\_type }\SpecialCharTok{==} \StringTok{"data\_sources"}
\NormalTok{)}

\CommentTok{\# Mostrar la información del proyecto seleccionado para confirmar que se ha filtrado correctamente}
\NormalTok{project\_info}
\end{Highlighting}
\end{Shaded}

\begin{verbatim}
##      project organism file_source     project_home project_type n_samples
## 57 SRP068565    human         sra data_sources/sra data_sources        20
\end{verbatim}

\end{document}
